\documentclass[]{article}
\usepackage{lmodern}
\usepackage{amssymb,amsmath}
\usepackage{ifxetex,ifluatex}
\usepackage{fixltx2e} % provides \textsubscript
\ifnum 0\ifxetex 1\fi\ifluatex 1\fi=0 % if pdftex
  \usepackage[T1]{fontenc}
  \usepackage[utf8]{inputenc}
\else % if luatex or xelatex
  \ifxetex
    \usepackage{mathspec}
  \else
    \usepackage{fontspec}
  \fi
  \defaultfontfeatures{Ligatures=TeX,Scale=MatchLowercase}
\fi
% use upquote if available, for straight quotes in verbatim environments
\IfFileExists{upquote.sty}{\usepackage{upquote}}{}
% use microtype if available
\IfFileExists{microtype.sty}{%
\usepackage{microtype}
\UseMicrotypeSet[protrusion]{basicmath} % disable protrusion for tt fonts
}{}
\usepackage[margin=1in]{geometry}
\usepackage{hyperref}
\hypersetup{unicode=true,
            pdftitle={RLab5 report},
            pdfauthor={Ryan Peters},
            pdfborder={0 0 0},
            breaklinks=true}
\urlstyle{same}  % don't use monospace font for urls
\usepackage{color}
\usepackage{fancyvrb}
\newcommand{\VerbBar}{|}
\newcommand{\VERB}{\Verb[commandchars=\\\{\}]}
\DefineVerbatimEnvironment{Highlighting}{Verbatim}{commandchars=\\\{\}}
% Add ',fontsize=\small' for more characters per line
\usepackage{framed}
\definecolor{shadecolor}{RGB}{248,248,248}
\newenvironment{Shaded}{\begin{snugshade}}{\end{snugshade}}
\newcommand{\KeywordTok}[1]{\textcolor[rgb]{0.13,0.29,0.53}{\textbf{#1}}}
\newcommand{\DataTypeTok}[1]{\textcolor[rgb]{0.13,0.29,0.53}{#1}}
\newcommand{\DecValTok}[1]{\textcolor[rgb]{0.00,0.00,0.81}{#1}}
\newcommand{\BaseNTok}[1]{\textcolor[rgb]{0.00,0.00,0.81}{#1}}
\newcommand{\FloatTok}[1]{\textcolor[rgb]{0.00,0.00,0.81}{#1}}
\newcommand{\ConstantTok}[1]{\textcolor[rgb]{0.00,0.00,0.00}{#1}}
\newcommand{\CharTok}[1]{\textcolor[rgb]{0.31,0.60,0.02}{#1}}
\newcommand{\SpecialCharTok}[1]{\textcolor[rgb]{0.00,0.00,0.00}{#1}}
\newcommand{\StringTok}[1]{\textcolor[rgb]{0.31,0.60,0.02}{#1}}
\newcommand{\VerbatimStringTok}[1]{\textcolor[rgb]{0.31,0.60,0.02}{#1}}
\newcommand{\SpecialStringTok}[1]{\textcolor[rgb]{0.31,0.60,0.02}{#1}}
\newcommand{\ImportTok}[1]{#1}
\newcommand{\CommentTok}[1]{\textcolor[rgb]{0.56,0.35,0.01}{\textit{#1}}}
\newcommand{\DocumentationTok}[1]{\textcolor[rgb]{0.56,0.35,0.01}{\textbf{\textit{#1}}}}
\newcommand{\AnnotationTok}[1]{\textcolor[rgb]{0.56,0.35,0.01}{\textbf{\textit{#1}}}}
\newcommand{\CommentVarTok}[1]{\textcolor[rgb]{0.56,0.35,0.01}{\textbf{\textit{#1}}}}
\newcommand{\OtherTok}[1]{\textcolor[rgb]{0.56,0.35,0.01}{#1}}
\newcommand{\FunctionTok}[1]{\textcolor[rgb]{0.00,0.00,0.00}{#1}}
\newcommand{\VariableTok}[1]{\textcolor[rgb]{0.00,0.00,0.00}{#1}}
\newcommand{\ControlFlowTok}[1]{\textcolor[rgb]{0.13,0.29,0.53}{\textbf{#1}}}
\newcommand{\OperatorTok}[1]{\textcolor[rgb]{0.81,0.36,0.00}{\textbf{#1}}}
\newcommand{\BuiltInTok}[1]{#1}
\newcommand{\ExtensionTok}[1]{#1}
\newcommand{\PreprocessorTok}[1]{\textcolor[rgb]{0.56,0.35,0.01}{\textit{#1}}}
\newcommand{\AttributeTok}[1]{\textcolor[rgb]{0.77,0.63,0.00}{#1}}
\newcommand{\RegionMarkerTok}[1]{#1}
\newcommand{\InformationTok}[1]{\textcolor[rgb]{0.56,0.35,0.01}{\textbf{\textit{#1}}}}
\newcommand{\WarningTok}[1]{\textcolor[rgb]{0.56,0.35,0.01}{\textbf{\textit{#1}}}}
\newcommand{\AlertTok}[1]{\textcolor[rgb]{0.94,0.16,0.16}{#1}}
\newcommand{\ErrorTok}[1]{\textcolor[rgb]{0.64,0.00,0.00}{\textbf{#1}}}
\newcommand{\NormalTok}[1]{#1}
\usepackage{graphicx,grffile}
\makeatletter
\def\maxwidth{\ifdim\Gin@nat@width>\linewidth\linewidth\else\Gin@nat@width\fi}
\def\maxheight{\ifdim\Gin@nat@height>\textheight\textheight\else\Gin@nat@height\fi}
\makeatother
% Scale images if necessary, so that they will not overflow the page
% margins by default, and it is still possible to overwrite the defaults
% using explicit options in \includegraphics[width, height, ...]{}
\setkeys{Gin}{width=\maxwidth,height=\maxheight,keepaspectratio}
\IfFileExists{parskip.sty}{%
\usepackage{parskip}
}{% else
\setlength{\parindent}{0pt}
\setlength{\parskip}{6pt plus 2pt minus 1pt}
}
\setlength{\emergencystretch}{3em}  % prevent overfull lines
\providecommand{\tightlist}{%
  \setlength{\itemsep}{0pt}\setlength{\parskip}{0pt}}
\setcounter{secnumdepth}{0}
% Redefines (sub)paragraphs to behave more like sections
\ifx\paragraph\undefined\else
\let\oldparagraph\paragraph
\renewcommand{\paragraph}[1]{\oldparagraph{#1}\mbox{}}
\fi
\ifx\subparagraph\undefined\else
\let\oldsubparagraph\subparagraph
\renewcommand{\subparagraph}[1]{\oldsubparagraph{#1}\mbox{}}
\fi

%%% Use protect on footnotes to avoid problems with footnotes in titles
\let\rmarkdownfootnote\footnote%
\def\footnote{\protect\rmarkdownfootnote}

%%% Change title format to be more compact
\usepackage{titling}

% Create subtitle command for use in maketitle
\newcommand{\subtitle}[1]{
  \posttitle{
    \begin{center}\large#1\end{center}
    }
}

\setlength{\droptitle}{-2em}

  \title{RLab5 report}
    \pretitle{\vspace{\droptitle}\centering\huge}
  \posttitle{\par}
    \author{Ryan Peters}
    \preauthor{\centering\large\emph}
  \postauthor{\par}
    \date{}
    \predate{}\postdate{}
  

\begin{document}
\maketitle

{
\setcounter{tocdepth}{2}
\tableofcontents
}
\section{Exercises}\label{exercises}

\begin{center}\rule{0.5\linewidth}{\linethickness}\end{center}

\begin{Shaded}
\begin{Highlighting}[]
\CommentTok{# pkgload is a custom script that's defined inside of the shared_lab_utils folder}
\KeywordTok{pkgload}\NormalTok{(}\KeywordTok{c}\NormalTok{(}\StringTok{"devtools"}\NormalTok{,}\StringTok{"crayon"}\NormalTok{,}\StringTok{"utils"}\NormalTok{))}
\end{Highlighting}
\end{Shaded}

\begin{verbatim}
## Loading required package: devtools
\end{verbatim}

\begin{verbatim}
## Loading required package: crayon
\end{verbatim}

\begin{Shaded}
\begin{Highlighting}[]
\CommentTok{# Instead of forcing the script to reload the dataframe every time we run it, }
\CommentTok{# lets do a quick check to see if it's already been instantiated from a }
\CommentTok{# previsou run first}
\NormalTok{df_name <-}\StringTok{ 'ames'}
\ControlFlowTok{if}\NormalTok{ (}\KeywordTok{exists}\NormalTok{(df_name)) }\KeywordTok{is.data.frame}\NormalTok{(}\KeywordTok{get}\NormalTok{(df_name)) }\ControlFlowTok{else}\NormalTok{\{}
  \KeywordTok{download.file}\NormalTok{(}\StringTok{'http://www.openintro.org/stat/data/ames.RData'}\NormalTok{, }\DataTypeTok{destfile =} \StringTok{'ames.RData'}\NormalTok{)}
  \KeywordTok{load}\NormalTok{(}\StringTok{'ames.RData'}\NormalTok{)}
\NormalTok{\}}
\end{Highlighting}
\end{Shaded}

\begin{Shaded}
\begin{Highlighting}[]
\NormalTok{area <-}\StringTok{ }\NormalTok{ames}\OperatorTok{$}\NormalTok{Gr.Liv.Area}
\NormalTok{price <-}\StringTok{ }\NormalTok{ames}\OperatorTok{$}\NormalTok{SalePrice}
\CommentTok{# getmode(vector) is a custom function defined in prep_script}
\NormalTok{area_mode <-}\StringTok{ }\NormalTok{area[}\KeywordTok{which.max}\NormalTok{(}\KeywordTok{tabulate}\NormalTok{(}\KeywordTok{match}\NormalTok{(}\KeywordTok{unique}\NormalTok{(area),area)))]}
\NormalTok{price_mode <-}\StringTok{ }\NormalTok{price[}\KeywordTok{which.max}\NormalTok{(}\KeywordTok{tabulate}\NormalTok{(}\KeywordTok{match}\NormalTok{(}\KeywordTok{unique}\NormalTok{(price),price)))]}
\end{Highlighting}
\end{Shaded}

\begin{Shaded}
\begin{Highlighting}[]
\NormalTok{sumry <-}\StringTok{ }\KeywordTok{summary}\NormalTok{(area)}
\NormalTok{sumry[}\StringTok{"mode"}\NormalTok{] <-}\StringTok{ }\NormalTok{area_mode}
\KeywordTok{print}\NormalTok{(sumry)}
\end{Highlighting}
\end{Shaded}

Min. 1st Qu. Median Mean 3rd Qu. Max. mode 334 1126 1442 1500 1743 5642
1656

\begin{Shaded}
\begin{Highlighting}[]
\KeywordTok{hist_with_labels}\NormalTok{(area, }\DataTypeTok{border=}\StringTok{"yellow"}\NormalTok{, }\DataTypeTok{col=}\StringTok{"dodgerblue"}\NormalTok{, }\DataTypeTok{las=}\DecValTok{1}\NormalTok{, }\DataTypeTok{xlab=}\StringTok{"area"}\NormalTok{)}
\end{Highlighting}
\end{Shaded}

List of 6 \$ breaks : num {[}1:13{]} 0 500 1000 1500 2000 2500 3000 3500
4000 4500 \ldots{} \$ counts : int {[}1:12{]} 6 433 1184 905 274 102 17
4 2 1 \ldots{} \$ density : num {[}1:12{]} 4.10e-06 2.96e-04 8.08e-04
6.18e-04 1.87e-04 \ldots{} \$ mids : num {[}1:12{]} 250 750 1250 1750
2250 2750 3250 3750 4250 4750 \ldots{} \$ xname : chr ``x'' \$ equidist:
logi TRUE - attr(*, ``class'')= chr ``histogram''
\includegraphics{Ryan_Peters_RLab5_report_files/figure-latex/area-histogam-1.pdf}

\subsubsection{Exercise 1:}\label{exercise-1}

\textbf{Describe this population distribution.}

This is a right-skewed (aka right-tailed) distribution, with a mean of
1500 ft\textsuperscript{2} and a median of 1442 ft\textsuperscript{2}.

Jump to \protect\hyperlink{top}{Table of Contents}

\begin{center}\rule{0.5\linewidth}{\linethickness}\end{center}

\begin{Shaded}
\begin{Highlighting}[]
\NormalTok{samp1 <-}\StringTok{ }\KeywordTok{sample}\NormalTok{(area, }\DecValTok{50}\NormalTok{)}
\KeywordTok{hist_with_labels}\NormalTok{(samp1, }\DataTypeTok{border=}\StringTok{"dodgerblue"}\NormalTok{, }\DataTypeTok{col=}\StringTok{"yellow"}\NormalTok{, }\DataTypeTok{las=}\DecValTok{1}\NormalTok{, }\DataTypeTok{xlab=}\StringTok{"area for sample population of 50"}\NormalTok{)}
\end{Highlighting}
\end{Shaded}

List of 6 \$ breaks : int {[}1:6{]} 500 1000 1500 2000 2500 3000 \$
counts : int {[}1:5{]} 7 13 22 5 3 \$ density : num {[}1:5{]} 0.00028
0.00052 0.00088 0.0002 0.00012 \$ mids : num {[}1:5{]} 750 1250 1750
2250 2750 \$ xname : chr ``x'' \$ equidist: logi TRUE - attr(*,
``class'')= chr ``histogram''
\includegraphics{Ryan_Peters_RLab5_report_files/figure-latex/samp1-1.pdf}

\subsubsection{Exercise 2:}\label{exercise-2}

\textbf{Describe the distribution of this sample. How does it compare to
the distribution of the population?}

It still appears to be right skewed, but it is definitly closer to a
normal distribution than the total population Jump to
\protect\hyperlink{top}{Table of Contents}

\begin{center}\rule{0.5\linewidth}{\linethickness}\end{center}

\begin{Shaded}
\begin{Highlighting}[]
\KeywordTok{mean}\NormalTok{(samp1)}
\end{Highlighting}
\end{Shaded}

{[}1{]} 1589.9

\begin{Shaded}
\begin{Highlighting}[]
\NormalTok{samp2 <-}\StringTok{ }\KeywordTok{sample}\NormalTok{(area, }\DecValTok{50}\NormalTok{)}
\KeywordTok{mean}\NormalTok{(samp2)}
\end{Highlighting}
\end{Shaded}

{[}1{]} 1533.2

\subsubsection{Exercise 3:}\label{exercise-3}

\textbf{Take a second sample, also of size 50, and call it samp2.}

\textbf{How does the mean of samp2 compare with the mean of samp1?}

It is definetly different, but it's hard to say if that's purely due to
chance or the distribution of the population.

\textbf{Suppose we took two more samples, one of size 100 and one of
size 1000. Which would you think would provide a more accurate estimate
of the population mean?}

Jump to \protect\hyperlink{top}{Table of Contents}

\begin{center}\rule{0.5\linewidth}{\linethickness}\end{center}

\begin{Shaded}
\begin{Highlighting}[]
\NormalTok{sample_means50 <-}\StringTok{ }\KeywordTok{rep}\NormalTok{(}\OtherTok{NA}\NormalTok{, }\DecValTok{5000}\NormalTok{)}

\ControlFlowTok{for}\NormalTok{(i }\ControlFlowTok{in} \DecValTok{1}\OperatorTok{:}\DecValTok{5000}\NormalTok{)\{}
\NormalTok{  samp <-}\StringTok{ }\KeywordTok{sample}\NormalTok{(area, }\DecValTok{50}\NormalTok{)}
\NormalTok{  sample_means50[i] <-}\StringTok{ }\KeywordTok{mean}\NormalTok{(samp)}
\NormalTok{\}}

\KeywordTok{hist_with_labels}\NormalTok{(sample_means50, }\DataTypeTok{text_col =} \StringTok{"black"}\NormalTok{, }\DataTypeTok{border=}\StringTok{"orange"}\NormalTok{, }\DataTypeTok{col=}\StringTok{"magenta3"}\NormalTok{, }\DataTypeTok{las=}\DecValTok{1}\NormalTok{, }\DataTypeTok{xlab=}\StringTok{"example means5 for sample populations of 50"}\NormalTok{)}
\end{Highlighting}
\end{Shaded}

List of 6 \$ breaks : int {[}1:12{]} 1250 1300 1350 1400 1450 1500 1550
1600 1650 1700 \ldots{} \$ counts : int {[}1:11{]} 6 57 321 824 1359
1276 757 296 81 18 \ldots{} \$ density : num {[}1:11{]} 0.000024
0.000228 0.001284 0.003296 0.005436 \ldots{} \$ mids : num {[}1:11{]}
1275 1325 1375 1425 1475 \ldots{} \$ xname : chr ``x'' \$ equidist: logi
TRUE - attr(*, ``class'')= chr ``histogram''
\includegraphics{Ryan_Peters_RLab5_report_files/figure-latex/unnamed-chunk-2-1.pdf}

\begin{Shaded}
\begin{Highlighting}[]
\KeywordTok{hist_with_labels}\NormalTok{(sample_means50, }\DataTypeTok{text_col =} \StringTok{"black"}\NormalTok{, }\DataTypeTok{border=}\StringTok{"orange3"}\NormalTok{, }\DataTypeTok{col=}\StringTok{"magenta3"}\NormalTok{, }\DataTypeTok{las=}\DecValTok{1}\NormalTok{,}\DataTypeTok{breaks =} \DecValTok{25}\NormalTok{)}
\end{Highlighting}
\end{Shaded}

List of 6 \$ breaks : int {[}1:12{]} 1250 1300 1350 1400 1450 1500 1550
1600 1650 1700 \ldots{} \$ counts : int {[}1:11{]} 6 57 321 824 1359
1276 757 296 81 18 \ldots{} \$ density : num {[}1:11{]} 0.000024
0.000228 0.001284 0.003296 0.005436 \ldots{} \$ mids : num {[}1:11{]}
1275 1325 1375 1425 1475 \ldots{} \$ xname : chr ``x'' \$ equidist: logi
TRUE - attr(*, ``class'')= chr ``histogram''
\includegraphics{Ryan_Peters_RLab5_report_files/figure-latex/unnamed-chunk-3-1.pdf}

\subsubsection{Exercise 4:}\label{exercise-4}

\textbf{How many elements are there in sample\_means50?}

\begin{Shaded}
\begin{Highlighting}[]
\KeywordTok{length}\NormalTok{(sample_means50)}
\end{Highlighting}
\end{Shaded}

{[}1{]} 5000 \textbf{Describe the sampling distribution, and be sure to
specifically note its center.}

It's an approximately normal distribution, with a mean around 1500.

\textbf{Would you expect the distribution to change if we instead
collected 50,000 sample means?}

I would expect the

Jump to \protect\hyperlink{top}{Table of Contents}

\begin{center}\rule{0.5\linewidth}{\linethickness}\end{center}

\begin{Shaded}
\begin{Highlighting}[]
\NormalTok{sample_means50 <-}\StringTok{ }\KeywordTok{rep}\NormalTok{(}\OtherTok{NA}\NormalTok{, }\DecValTok{5000}\NormalTok{)}

\NormalTok{samp <-}\StringTok{ }\KeywordTok{sample}\NormalTok{(area, }\DecValTok{50}\NormalTok{)}
\NormalTok{sample_means50[}\DecValTok{1}\NormalTok{] <-}\StringTok{ }\KeywordTok{mean}\NormalTok{(samp)}

\NormalTok{samp <-}\StringTok{ }\KeywordTok{sample}\NormalTok{(area, }\DecValTok{50}\NormalTok{)}
\NormalTok{sample_means50[}\DecValTok{2}\NormalTok{] <-}\StringTok{ }\KeywordTok{mean}\NormalTok{(samp)}

\NormalTok{samp <-}\StringTok{ }\KeywordTok{sample}\NormalTok{(area, }\DecValTok{50}\NormalTok{)}
\NormalTok{sample_means50[}\DecValTok{3}\NormalTok{] <-}\StringTok{ }\KeywordTok{mean}\NormalTok{(samp)}

\NormalTok{samp <-}\StringTok{ }\KeywordTok{sample}\NormalTok{(area, }\DecValTok{50}\NormalTok{)}
\NormalTok{sample_means50[}\DecValTok{4}\NormalTok{] <-}\StringTok{ }\KeywordTok{mean}\NormalTok{(samp)}
\end{Highlighting}
\end{Shaded}

\begin{Shaded}
\begin{Highlighting}[]
\NormalTok{sample_means50 <-}\StringTok{ }\KeywordTok{rep}\NormalTok{(}\OtherTok{NA}\NormalTok{, }\DecValTok{5000}\NormalTok{)}

\ControlFlowTok{for}\NormalTok{(i }\ControlFlowTok{in} \DecValTok{1}\OperatorTok{:}\DecValTok{5000}\NormalTok{)\{}
\NormalTok{  samp <-}\StringTok{ }\KeywordTok{sample}\NormalTok{(area, }\DecValTok{50}\NormalTok{)}
\NormalTok{  sample_means50[i] <-}\StringTok{ }\KeywordTok{mean}\NormalTok{(samp)}
  \CommentTok{# print(i)}
\NormalTok{\}}
\end{Highlighting}
\end{Shaded}

\subsubsection{Exercise 5:}\label{exercise-5}

\textbf{To make sure you understand what you've done in this loop, try
running a smaller version. Initialize a vector of 100 zeros called
sample\_means\_small. Run a loop that takes a sample of size 50 from
area and stores the sample mean in sample\_means\_small, but only
iterate from 1 to 100. Print the output to your screen (type
sample\_means\_small into the console and press enter). How many
elements are there in this object called sample\_means\_small? What does
each element represent?}

Jump to \protect\hyperlink{top}{Table of Contents}

\begin{center}\rule{0.5\linewidth}{\linethickness}\end{center}

\begin{Shaded}
\begin{Highlighting}[]
\KeywordTok{hist_with_labels}\NormalTok{(sample_means50, }\DataTypeTok{text_col =} \StringTok{"black"}\NormalTok{, }\DataTypeTok{border=}\StringTok{"red2"}\NormalTok{, }\DataTypeTok{col=}\StringTok{"cyan"}\NormalTok{, }\DataTypeTok{las=}\DecValTok{1}\NormalTok{)}
\end{Highlighting}
\end{Shaded}

List of 6 \$ breaks : int {[}1:13{]} 1250 1300 1350 1400 1450 1500 1550
1600 1650 1700 \ldots{} \$ counts : int {[}1:12{]} 7 60 312 893 1303
1232 758 326 91 15 \ldots{} \$ density : num {[}1:12{]} 0.000028 0.00024
0.001248 0.003572 0.005212 \ldots{} \$ mids : num {[}1:12{]} 1275 1325
1375 1425 1475 \ldots{} \$ xname : chr ``x'' \$ equidist: logi TRUE -
attr(*, ``class'')= chr ``histogram''
\includegraphics{Ryan_Peters_RLab5_report_files/figure-latex/sample_means50-1.pdf}

\begin{Shaded}
\begin{Highlighting}[]
\NormalTok{sample_means10 <-}\StringTok{ }\KeywordTok{rep}\NormalTok{(}\OtherTok{NA}\NormalTok{, }\DecValTok{5000}\NormalTok{)}
\NormalTok{sample_means100 <-}\StringTok{ }\KeywordTok{rep}\NormalTok{(}\OtherTok{NA}\NormalTok{, }\DecValTok{5000}\NormalTok{)}

\ControlFlowTok{for}\NormalTok{(i }\ControlFlowTok{in} \DecValTok{1}\OperatorTok{:}\DecValTok{5000}\NormalTok{)\{}
\NormalTok{  samp <-}\StringTok{ }\KeywordTok{sample}\NormalTok{(area, }\DecValTok{10}\NormalTok{)}
\NormalTok{  sample_means10[i] <-}\StringTok{ }\KeywordTok{mean}\NormalTok{(samp)}
\NormalTok{  samp <-}\StringTok{ }\KeywordTok{sample}\NormalTok{(area, }\DecValTok{100}\NormalTok{)}
\NormalTok{  sample_means100[i] <-}\StringTok{ }\KeywordTok{mean}\NormalTok{(samp)}
\NormalTok{\}}
\end{Highlighting}
\end{Shaded}

\begin{Shaded}
\begin{Highlighting}[]
\KeywordTok{par}\NormalTok{(}\DataTypeTok{mfrow =} \KeywordTok{c}\NormalTok{(}\DecValTok{3}\NormalTok{, }\DecValTok{1}\NormalTok{),}
    \DataTypeTok{oma =} \KeywordTok{c}\NormalTok{(}\DecValTok{0}\NormalTok{,}\DecValTok{0}\NormalTok{,}\DecValTok{0}\NormalTok{,}\DecValTok{0}\NormalTok{) }\OperatorTok{+}\StringTok{ }\DecValTok{1}\NormalTok{,}
    \DataTypeTok{mar =} \KeywordTok{c}\NormalTok{(}\DecValTok{0}\NormalTok{,}\DecValTok{1}\NormalTok{,}\DecValTok{2}\NormalTok{,}\DecValTok{1}\NormalTok{) }\OperatorTok{+}\StringTok{ }\DecValTok{1}\NormalTok{)}

\NormalTok{xlimits <-}\StringTok{ }\KeywordTok{range}\NormalTok{(sample_means10)}

\CommentTok{# hist(sample_means10, breaks = 20, xlim = xlimits)}
\KeywordTok{hist_with_labels}\NormalTok{(sample_means10, }\DataTypeTok{text_col =} \StringTok{"yellow"}\NormalTok{, }\DataTypeTok{breaks =} \DecValTok{20}\NormalTok{, }\DataTypeTok{xlim =}\NormalTok{ xlimits, }\DataTypeTok{border=}\StringTok{"gray"}\NormalTok{, }\DataTypeTok{col=}\StringTok{"dodgerblue"}\NormalTok{, }\DataTypeTok{las=}\DecValTok{1}\NormalTok{)}
\end{Highlighting}
\end{Shaded}

List of 6 \$ breaks : int {[}1:15{]} 900 1000 1100 1200 1300 1400 1500
1600 1700 1800 \ldots{} \$ counts : int {[}1:14{]} 1 9 77 411 885 1298
1094 699 334 128 \ldots{} \$ density : num {[}1:14{]} 0.000002 0.000018
0.000154 0.000822 0.00177 \ldots{} \$ mids : num {[}1:14{]} 950 1050
1150 1250 1350 1450 1550 1650 1750 1850 \ldots{} \$ xname : chr ``x'' \$
equidist: logi TRUE - attr(*, ``class'')= chr ``histogram''

\begin{Shaded}
\begin{Highlighting}[]
\CommentTok{# hist(sample_means50, breaks = 20, xlim = xlimits)}
\KeywordTok{hist_with_labels}\NormalTok{(sample_means50, }\DataTypeTok{text_col =} \StringTok{"yellow"}\NormalTok{, }\DataTypeTok{border=}\StringTok{"gray"}\NormalTok{, }\DataTypeTok{col=}\StringTok{"dodgerblue"}\NormalTok{, }\DataTypeTok{las=}\DecValTok{1}\NormalTok{)}
\end{Highlighting}
\end{Shaded}

List of 6 \$ breaks : int {[}1:13{]} 1250 1300 1350 1400 1450 1500 1550
1600 1650 1700 \ldots{} \$ counts : int {[}1:12{]} 7 60 312 893 1303
1232 758 326 91 15 \ldots{} \$ density : num {[}1:12{]} 0.000028 0.00024
0.001248 0.003572 0.005212 \ldots{} \$ mids : num {[}1:12{]} 1275 1325
1375 1425 1475 \ldots{} \$ xname : chr ``x'' \$ equidist: logi TRUE -
attr(*, ``class'')= chr ``histogram''

\begin{Shaded}
\begin{Highlighting}[]
\CommentTok{# hist(sample_means100, breaks = 20, xlim = xlimits)}
\KeywordTok{hist_with_labels}\NormalTok{(sample_means100, }\DataTypeTok{text_col =} \StringTok{"yellow"}\NormalTok{, }\DataTypeTok{breaks =} \DecValTok{20}\NormalTok{, }\DataTypeTok{xlim =}\NormalTok{ xlimits, }\DataTypeTok{border=}\StringTok{"gray"}\NormalTok{, }\DataTypeTok{col=}\StringTok{"dodgerblue"}\NormalTok{, }\DataTypeTok{las=}\DecValTok{1}\NormalTok{)}
\end{Highlighting}
\end{Shaded}

List of 6 \$ breaks : int {[}1:21{]} 1320 1340 1360 1380 1400 1420 1440
1460 1480 1500 \ldots{} \$ counts : int {[}1:20{]} 2 6 16 84 148 333 503
695 803 788 \ldots{} \$ density : num {[}1:20{]} 0.00002 0.00006 0.00016
0.00084 0.00148 0.00333 0.00503 0.00695 0.00803 0.00788 \ldots{} \$ mids
: num {[}1:20{]} 1330 1350 1370 1390 1410 1430 1450 1470 1490 1510
\ldots{} \$ xname : chr ``x'' \$ equidist: logi TRUE - attr(*,
``class'')= chr ``histogram''
\includegraphics{Ryan_Peters_RLab5_report_files/figure-latex/sample_means10-1.pdf}

\subsubsection{Exercise 6:}\label{exercise-6}

\textbf{When the sample size is larger, what happens to the center? What
about the spread?}

Jump to \protect\hyperlink{top}{Table of Contents}

=====

\section{On your own}\label{on-your-own}

So far, we have only focused on estimating the mean living area in homes
in Ames. Now you'll try to estimate the mean home price.

\subsubsection{On Your Own 1:}\label{on-your-own-1}

\textbf{Take a random sample of size 50 from price. Using this sample,
what is your best point estimate of the population mean?}

\begin{Shaded}
\begin{Highlighting}[]
\NormalTok{price_sample <-}\StringTok{ }\KeywordTok{sample}\NormalTok{(price, }\DecValTok{50}\NormalTok{)}
\KeywordTok{hist_with_labels}\NormalTok{(price_sample, }
                 \DataTypeTok{border=}\StringTok{"white"}\NormalTok{, }
                 \DataTypeTok{col=}\StringTok{"blue1"}\NormalTok{, }
                 \DataTypeTok{las=}\DecValTok{1}\NormalTok{, }
                 \DataTypeTok{xlab=}\StringTok{"property prices for sample population of size = 50"}\NormalTok{,}
                 \DataTypeTok{main=}\StringTok{"Histogram for}\CharTok{\textbackslash{}n}\StringTok{single sample population of 50 samples"}\NormalTok{)}
\end{Highlighting}
\end{Shaded}

List of 6 \$ breaks : int {[}1:7{]} 50000 100000 150000 200000 250000
300000 350000 \$ counts : int {[}1:6{]} 4 16 16 8 4 2 \$ density : num
{[}1:6{]} 1.6e-06 6.4e-06 6.4e-06 3.2e-06 1.6e-06 8.0e-07 \$ mids : num
{[}1:6{]} 75000 125000 175000 225000 275000 325000 \$ xname : chr ``x''
\$ equidist: logi TRUE - attr(*, ``class'')= chr ``histogram''
\includegraphics{Ryan_Peters_RLab5_report_files/figure-latex/unnamed-chunk-7-1.pdf}

\begin{Shaded}
\begin{Highlighting}[]
\NormalTok{smry <-}\StringTok{ }\KeywordTok{summary}\NormalTok{(price_sample)}
\KeywordTok{cat}\NormalTok{(}\StringTok{"My best point estimate for the population mean is the sample mean: ***"}\NormalTok{,smry[[}\DecValTok{4}\NormalTok{]],}\StringTok{"***}\CharTok{\textbackslash{}n}\StringTok{"}\NormalTok{,}\DataTypeTok{sep=}\StringTok{""}\NormalTok{)}
\end{Highlighting}
\end{Shaded}

My best point estimate for the population mean is the sample mean:
\textbf{\emph{173174.6}}

Jump to \protect\hyperlink{top}{Table of Contents}

\subsubsection{On Your Own 2:}\label{on-your-own-2}

\textbf{Since you have access to the population, simulate the sampling
distribution for x-bar for price data by taking 5000 samples from the
population of size 50 and computing 5000 sample means.}

\textbf{Store these means in a vector called sample\_means50.}

\textbf{Plot the data, then describe the shape of this sampling
distribution.}

\textbf{Based on this sampling distribution, what would you guess the
mean home price of the population to be?}

\textbf{Finally, calculate and report the population mean.}

\begin{Shaded}
\begin{Highlighting}[]
\NormalTok{ sample_means50 <-}\StringTok{ }\KeywordTok{range}\NormalTok{(}\DecValTok{5000}\NormalTok{)}
\ControlFlowTok{for}\NormalTok{(i }\ControlFlowTok{in} \DecValTok{0}\OperatorTok{:}\DecValTok{5000}\NormalTok{)\{}
\NormalTok{   sample_means50[i] <-}\StringTok{ }\KeywordTok{mean}\NormalTok{(}\KeywordTok{sample}\NormalTok{(price,}\DecValTok{50}\NormalTok{))}
\NormalTok{\}}

\KeywordTok{hist_with_labels}\NormalTok{( sample_means50, }
                  \DataTypeTok{border=}\StringTok{"white"}\NormalTok{, }
                  \DataTypeTok{col=}\StringTok{"blue1"}\NormalTok{, }
                  \DataTypeTok{las=}\DecValTok{1}\NormalTok{, }
                  \DataTypeTok{xlab=} \StringTok{"property value averages from sample populations of size = 50"}\NormalTok{,}
                  \DataTypeTok{main=}\StringTok{"Histogram of means for}\CharTok{\textbackslash{}n}\StringTok{50k sample populations of 50 samples a pieace"}\NormalTok{)}
\end{Highlighting}
\end{Shaded}

List of 6 \$ breaks : int {[}1:18{]} 140000 145000 150000 155000 160000
165000 170000 175000 180000 185000 \ldots{} \$ counts : int {[}1:17{]} 1
3 16 68 247 473 676 878 904 706 \ldots{} \$ density : num {[}1:17{]}
4.00e-08 1.20e-07 6.40e-07 2.72e-06 9.88e-06 \ldots{} \$ mids : num
{[}1:17{]} 142500 147500 152500 157500 162500 \ldots{} \$ xname : chr
``x'' \$ equidist: logi TRUE - attr(*, ``class'')= chr ``histogram''
\includegraphics{Ryan_Peters_RLab5_report_files/figure-latex/unnamed-chunk-8-1.pdf}

\begin{Shaded}
\begin{Highlighting}[]
\NormalTok{smry_50k <-}\StringTok{ }\KeywordTok{summary}\NormalTok{( sample_means50)}

\KeywordTok{cat}\NormalTok{(}\StringTok{"The distribution is approximately normal, maybe slightly skewed to the rigt still}\CharTok{\textbackslash{}n}\StringTok{From the histogram, I would estimate the mean home price to be approximately $180k}\CharTok{\textbackslash{}n}\StringTok{The calculated mean of 50k means on sample populations of 50... is: ***"}\NormalTok{,smry_50k[[}\DecValTok{4}\NormalTok{]],}\StringTok{"***}\CharTok{\textbackslash{}n}\StringTok{"}\NormalTok{,}\DataTypeTok{sep=}\StringTok{""}\NormalTok{)}
\end{Highlighting}
\end{Shaded}

The distribution is approximately normal, maybe slightly skewed to the
rigt still From the histogram, I would estimate the mean home price to
be approximately \$180k The calculated mean of 50k means on sample
populations of 50\ldots{} is: \textbf{\emph{181119.9}}

Jump to \protect\hyperlink{top}{Table of Contents}

\subsubsection{On Your Own 3:}\label{on-your-own-3}

\textbf{Change your sample size from 50 to 150, then compute the
sampling distribution using the same method as above, and store these
means in a new vector called sample\_means150.}

\textbf{Describe the shape of this sampling distribution, and compare it
to the sampling distribution for a sample size of 50.}

\textbf{Based on this sampling distribution, what would you guess to be
the meansale price of homes in Ames?}

\begin{Shaded}
\begin{Highlighting}[]
\NormalTok{sample_means150 <-}\StringTok{ }\KeywordTok{range}\NormalTok{(}\DecValTok{5000}\NormalTok{)}
\ControlFlowTok{for}\NormalTok{(i }\ControlFlowTok{in} \DecValTok{0}\OperatorTok{:}\DecValTok{5000}\NormalTok{)\{}
\NormalTok{   sample_means150[i] <-}\StringTok{ }\KeywordTok{mean}\NormalTok{(}\KeywordTok{sample}\NormalTok{(price,}\DecValTok{150}\NormalTok{))}
\NormalTok{\}}

\KeywordTok{hist_with_labels}\NormalTok{( sample_means150, }
                  \DataTypeTok{border=}\StringTok{"white"}\NormalTok{, }
                  \DataTypeTok{col=}\StringTok{"blue1"}\NormalTok{, }
                  \DataTypeTok{las=}\DecValTok{1}\NormalTok{, }
                  \DataTypeTok{xlab=} \StringTok{"property value averages from sample populations of size = 150"}\NormalTok{,}
                  \DataTypeTok{main=}\StringTok{"Histogram of means for}\CharTok{\textbackslash{}n}\StringTok{50k sample populations of 150 samples a pieace"}\NormalTok{)}
\end{Highlighting}
\end{Shaded}

List of 6 \$ breaks : int {[}1:11{]} 155000 160000 165000 170000 175000
180000 185000 190000 195000 200000 \ldots{} \$ counts : int {[}1:10{]} 2
16 190 724 1409 1426 853 309 63 8 \$ density : num {[}1:10{]} 8.00e-08
6.40e-07 7.60e-06 2.90e-05 5.64e-05 \ldots{} \$ mids : num {[}1:10{]}
157500 162500 167500 172500 177500 \ldots{} \$ xname : chr ``x'' \$
equidist: logi TRUE - attr(*, ``class'')= chr ``histogram''
\includegraphics{Ryan_Peters_RLab5_report_files/figure-latex/unnamed-chunk-9-1.pdf}

\begin{Shaded}
\begin{Highlighting}[]
\NormalTok{smry_150k <-}\StringTok{ }\KeywordTok{summary}\NormalTok{( sample_means150)}

\KeywordTok{cat}\NormalTok{(}\StringTok{"The distribution is approximately normal, though it is much less skewed than the sample size of only 50}\CharTok{\textbackslash{}n}\StringTok{From the histogram, I would estimate the mean home price to be approximately $182k}\CharTok{\textbackslash{}n}\StringTok{The calculated mean of 50k means  on sample populations of 150... is: ***"}\NormalTok{,smry_150k[[}\DecValTok{4}\NormalTok{]],}\StringTok{"***}\CharTok{\textbackslash{}n}\StringTok{"}\NormalTok{,}\DataTypeTok{sep=}\StringTok{""}\NormalTok{)}
\end{Highlighting}
\end{Shaded}

The distribution is approximately normal, though it is much less skewed
than the sample size of only 50 From the histogram, I would estimate the
mean home price to be approximately \$182k The calculated mean of 50k
means on sample populations of 150\ldots{} is: \textbf{\emph{180692.6}}

Jump to \protect\hyperlink{top}{Table of Contents}

\subsubsection{On Your Own 4:}\label{on-your-own-4}

\textbf{Of the sampling distributions from 2 and 3, which has a smaller
spread?}

\begin{Shaded}
\begin{Highlighting}[]
\KeywordTok{cat}\NormalTok{(}\StringTok{"The larger sample population has the smaller spread"}\NormalTok{)}
\end{Highlighting}
\end{Shaded}

The larger sample population has the smaller spread \textbf{If we're
concerned with making estimates that are more often close to the true
value, would we prefer a distribution with a large or small spread?}

\begin{Shaded}
\begin{Highlighting}[]
\NormalTok{price_summary <-}\StringTok{ }\KeywordTok{summary}\NormalTok{(price)}
\NormalTok{true_mean <-}\StringTok{ }\NormalTok{price_summary[[}\DecValTok{4}\NormalTok{]]}
\NormalTok{winner_winner_chicken_dinner =}\StringTok{ "Neither sample size has a clear advantage"}
\ControlFlowTok{if}\NormalTok{(}\KeywordTok{abs}\NormalTok{(smry_50k[[}\DecValTok{4}\NormalTok{]]}\OperatorTok{-}\NormalTok{true_mean) }\OperatorTok{<}\StringTok{ }\KeywordTok{abs}\NormalTok{(smry_150k[[}\DecValTok{4}\NormalTok{]]}\OperatorTok{-}\NormalTok{true_mean))\{}
  \CommentTok{# the smaller sample size is closer to the true mean}
\NormalTok{  winner_winner_chicken_dinner =}\StringTok{ "The smaller sample size appears to yield results closer to the population parameter mean"}
\NormalTok{\} }\ControlFlowTok{else} \ControlFlowTok{if}\NormalTok{(}\KeywordTok{abs}\NormalTok{(smry_50k[[}\DecValTok{4}\NormalTok{]]}\OperatorTok{-}\NormalTok{true_mean) }\OperatorTok{<}\StringTok{ }\KeywordTok{abs}\NormalTok{(smry_150k[[}\DecValTok{4}\NormalTok{]]}\OperatorTok{-}\NormalTok{true_mean))\{}
  \CommentTok{# the larger sample size is closer to the true mean}
\NormalTok{  winner_winner_chicken_dinner =}\StringTok{ "The larger sample size appears to yield results closer to the population parameter mean"}
\NormalTok{\}}
\end{Highlighting}
\end{Shaded}

I believe we would want a distribution with a narrower spread as it will
more consistently provide an approximation that is reliably close to the
true mean.

After comparing samples against the true mean, it appears that the
results are: \emph{Neither sample size has a clear advantage}

Jump to \protect\hyperlink{top}{Table of Contents}

\section{Disclaimers and such}\label{disclaimers-and-such}

This is a product of OpenIntro that is released under a Creative Commons
Attribution-ShareAlike 3.0 Unported.

This lab was written for OpenIntro by Andrew Bray and Mine
Çetinkaya-Rundel.


\end{document}
